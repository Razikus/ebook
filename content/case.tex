% Chapter 
\chapter{Select case}

\lettrine{Y}{our PCB is validated} and software development is underway. Now comes a critical decision that affects manufacturing cost, assembly complexity, and market perception: selecting or designing the enclosure.

The case decision impacts every aspect of production - from assembly time to shipping costs to customer perception of quality. Choose wrong and you'll face expensive redesigns or manufacturing complications.

\section{Standard vs. Custom Enclosure Decision}

\subsection{Existing Standard Cases}

\textbf{Kradex and established manufacturers offer proven solutions:}
\begin{itemize}
\item Immediate availability with known dimensions
\item Lower upfront cost - no tooling investment
\item Proven durability and regulatory compliance
\item Multiple mounting options typically available
\item Standardized assembly processes
\end{itemize}

\textbf{When to choose standard cases:}
\begin{itemize}
\item Production volume under 500 units
\item PCB can be designed to fit standard dimensions
\item No unusual interface or mounting requirements
\item Fast time-to-market priority
\item Limited development budget
\end{itemize}

\subsection{Custom Case Development}

\textbf{Design your own enclosure when:}
\begin{itemize}
\item Unique form factor provides competitive advantage
\item Standard cases don't accommodate your interfaces
\item Production volume justifies tooling investment
\item Brand differentiation requires custom appearance
\end{itemize}

\section{3D Printing Technology Selection}

\subsection{The Professional Standard: MJF}

\textbf{Multi Jet Fusion (MJF) is the only professional 3D printing option.}

Avoid these inferior technologies:
\begin{itemize}
\item \textbf{FDM/FFF:} Layer lines visible, poor surface finish, mechanical weakness
\item \textbf{Resin printing:} Toxic materials, poor durability, surface quality issues
\item \textbf{SLA:} Expensive, limited materials, post-processing complexity
\end{itemize}

\textbf{MJF advantages for production prototypes:}
\begin{itemize}
\item Professional surface finish straight from printer
\item Excellent dimensional accuracy and repeatability
\item Strong, functional parts suitable for real-world testing
\item No visible layer lines or support marks
\item Cost-effective for complex geometries
\end{itemize}

\textbf{Real MJF prototype costs (Cubic Inch):}

\begin{table}[h]
\centering
\begin{tabular}{|l|c|c|}
\hline
\textbf{Part} & \textbf{Quantity} & \textbf{Cost (EUR)} \\
\hline
Bottom enclosure & 2 units & €38 \\
Top cover & 2 units & €9 \\
Shipping & - & €11 \\
VAT & - & €10 \\
\hline
\textbf{Total order} & \textbf{4 parts} & \textbf{€100} \\
\hline
\end{tabular}
\caption{MJF prototype costs - approximately €50 per complete enclosure including delivery}
\end{table}

Material: HP 3D High Reusability PA 12 (Nylon)
Tolerance: ±0.3mm
Delivery: 6 working days
Finish: Dye Black

\section{Manufacturing Progression Strategy}

\subsection{Three-Stage Development Path}

\textbf{Stage 1: Initial Prototyping (1-5 units)}
Use any available 3D printing technology for basic fit testing and PCB mounting validation. Quality is secondary to speed and cost at this stage.

\textbf{Stage 2: Functional Prototypes (5-50 units)}
Switch to MJF printing for professional appearance and mechanical properties. Use these units for customer demonstrations and extended testing.

\textbf{Stage 3: Production Decision Point (50+ units)}
At 50 units sold, evaluate injection molding. The break-even point varies by complexity, but tooling costs typically amortize around 100-500 units.

\section{Mounting Strategy}

\subsection{Installation Method Planning}

\textbf{Wall mounting considerations:}
\begin{itemize}
\item Standard wall anchor compatibility
\item Cable management and strain relief
\item Access to status indicators and interfaces
\item Theft resistance and tamper detection
\item Level mounting with adjustment capability
\end{itemize}

\textbf{Desktop device requirements:}
\begin{itemize}
\item Stable base preventing tip-over
\item Non-slip feet or weighted base
\item Cable routing that doesn't interfere with stability
\item User interface accessibility and viewing angles
\end{itemize}

\subsection{PCB Mounting Methodology}

\textbf{Professional PCB mounting requires precise measurements:}
\begin{itemize}
\item Use calipers to measure exact PCB dimensions
\item Account for component clearance above PCB surface
\item Include tolerance for manufacturing variations (±0.2mm typical)
\item Verify mounting hole positions with multiple PCB samples
\item Design for thermal expansion if operating temperatures vary
\end{itemize}

\begin{tcolorbox}[colback=blue!10,colframe=blue!75!black,title=Alternative: Hot Glue Mounting]
For non-heating devices and low-volume production, hot glue provides acceptable PCB mounting. This eliminates precision machining requirements and accommodates PCB variations. Only suitable when device temperature remains below 60°C.
\end{tcolorbox}

\section{Injection Molding Considerations}

\subsection{Always Design with Injection Molding in Mind}

Even if starting with 3D printing, design case geometry compatible with injection molding from the beginning. Redesigning for injection molding later is expensive and time-consuming.

\textbf{Injection mold-friendly design principles:}
\begin{itemize}
\item \textbf{Keep it simple:} Minimize complex geometries and undercuts
\item \textbf{Uniform wall thickness:} Typically 1.5-3mm for electronics enclosures
\item \textbf{Draft angles:} 1-2 degrees minimum on all vertical surfaces
\item \textbf{Avoid thin features:} Minimum feature size depends on material choice
\item \textbf{Split line planning:} Consider where mold halves separate
\end{itemize}

\subsection{Injection Molding Timeline}

\textbf{Recommended progression:}
\begin{enumerate}
\item \textbf{1-5 units:} Any 3D printing for basic validation
\item \textbf{5-50 units:} MJF printing for professional appearance
\item \textbf{50+ units sold:} Evaluate injection molding economics
\item \textbf{1000+ units planned:} Injection molding becomes cost-effective
\end{enumerate}

\textbf{Fullbax Formy partnership} provides established China manufacturing relationships, reducing risk and complexity for European companies entering injection molding.

\section{Design Simplicity Imperative}

\subsection{Complexity Kills Profitability}

Every additional feature, curve, or detail increases manufacturing cost and complexity. Optimize for manufacturability, not aesthetic sophistication.

\textbf{Cost-driving complexity factors:}
\begin{itemize}
\item Multiple split lines requiring precision alignment
\item Threaded inserts or complex mechanical features
\item Multiple materials or overmolding requirements
\item Tight tolerances beyond standard manufacturing capability
\item Surface textures or decorative elements
\end{itemize}

\textbf{Simplicity benefits:}
\begin{itemize}
\item Lower tooling costs and faster delivery
\item Reduced assembly time and labor costs
\item Fewer quality control points and failure modes
\item Easier inventory management and logistics
\item Simplified service and repair procedures
\end{itemize}

\section{Case Selection Workflow}

\subsection{Systematic Decision Process}

\begin{enumerate}
\item \textbf{Define requirements:} Interface access, mounting method, environmental conditions
\item \textbf{Survey standard options:} Evaluate Kradex and similar catalogs
\item \textbf{PCB fit analysis:} Can board be designed to fit standard enclosure?
\item \textbf{Cost comparison:} Standard case + PCB redesign vs. custom tooling
\item \textbf{Volume projection:} Realistic sales forecast over 2-year period
\item \textbf{Prototype with cheapest option:} Validate assumptions before committing
\item \textbf{Scale manufacturing method:} Progress through 3D printing to injection molding
\end{enumerate}

\section{Quality and Durability Testing}

\subsection{Enclosure Validation Requirements}

\textbf{Mechanical testing:}
\begin{itemize}
\item Drop testing from 1-meter height (simulate installation accidents)
\item Connector insertion/removal cycling (500+ cycles minimum)
\item Environmental stress testing (temperature, humidity cycling)
\item UV exposure testing for outdoor or window-mounted devices
\end{itemize}

\textbf{Thermal performance:}
\begin{itemize}
\item Internal temperature monitoring during extended operation
\item Hot spot identification using thermal imaging
\item Ventilation effectiveness measurement
\item Component derating analysis based on actual temperatures
\end{itemize}

\section{Common Case Selection Mistakes}

\textbf{Overengineering aesthetic appeal:} Complex curves and features dramatically increase cost without improving functionality.

\textbf{Ignoring assembly labor:} Cases requiring extensive manual assembly increase per-unit cost and quality variation.

\textbf{Premature injection molding:} Committing to tooling before validating market demand and design stability.

\textbf{Inadequate thermal planning:} Failing to account for heat generation in enclosed operation.

\textbf{Poor cable management:} Not planning strain relief and connector access during case design.

Remember: the best case is the simplest one that meets functional requirements and scales cost-effectively with production volume. Aesthetic sophistication should never compromise manufacturability or profitability.