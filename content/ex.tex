% Chapter XIII
\chapter{Repeat}

\lettrine{S}{caling from your first} successful product to a sustainable business requires systematic repetition of proven processes. After shipping your first 100 boards, the temptation is to celebrate and move on to the next idea. Resist this urge.

The real profit comes from iteration and refinement of existing products. Each production run teaches you something new about manufacturing, quality control, and customer needs.

\section{The Iteration Mindset}

Every production run is an experiment. Document what works and what doesn't. Your goal is to reduce the time between identifying a market need and delivering a solution.

\begin{quote}
\textit{"The first version of your product will be wrong. The second version will be less wrong. By the tenth version, you might actually have something customers want to pay for."} — Manufacturing wisdom
\end{quote}

\section{Version Control for Hardware}

Unlike software, hardware versions are expensive. Each PCB revision costs money and time. Establish clear criteria for when to rev your board:

\begin{itemize}
\item Critical bugs that affect more than 5\% of units
\item Component obsolescence forcing a redesign
\item Cost reduction opportunities greater than 15\%
\item Customer-requested features with proven demand
\end{itemize}

\section{Production Checklist}

Before each production run, verify these items:

\begin{enumerate}
\item \textbf{Component availability} — Check lead times for all parts
\item \textbf{Assembly house capacity} — Confirm production slots
\item \textbf{Test procedures} — Update based on field failures
\item \textbf{Packaging} — Ensure adequate protection for shipping
\item \textbf{Documentation} — Update assembly drawings and BOMs
\end{enumerate}

\section{Common Pitfalls}

\subsection{The Feature Creep Trap}

Customers will request endless features. Not all feedback is worth implementing. Ask yourself:
\begin{itemize}
\item Does this solve a real problem for multiple customers?
\item Can we implement it without major redesign?
\item Will this feature increase or decrease reliability?
\end{itemize}

\subsection{Cost Optimization Timing}

Don't optimize costs too early. Wait until you have:
\begin{itemize}
\item Stable hardware design (no major revisions for 6 months)
\item Predictable demand (at least 3 consecutive successful production runs)
\item Clear understanding of failure modes
\end{itemize}

\section{Code Quality in Production}

Your software must be bulletproof. Here's a minimal checklist for production code:

\begin{tcolorbox}[colback=gray!5,colframe=gray!50,title=Production Validation Script]
\begin{lstlisting}[
  language=bash,
  basicstyle=\small\ttfamily,
  keywordstyle=\color{blue}\bfseries,
  commentstyle=\color{green!50!black}\itshape,
  stringstyle=\color{red},
  showstringspaces=false,
  breaklines=true,
  numbers=left,
  numberstyle=\tiny\color{gray},
  frame=none
]
#!/bin/bash
# Pre-deployment checklist
echo "Running production code validation..."

# Check for debug statements
if grep -r "console.log|print|debug" src/; then
    echo "ERROR: Debug statements found!"
    exit 1
fi

# Verify error handling
if ! grep -r "try|catch|except" src/; then
    echo "WARNING: No error handling found"
fi

# Check for hardcoded values
if grep -r "192.168|localhost|127.0.0.1" src/; then
    echo "ERROR: Hardcoded network values found!"
    exit 1
fi

echo "Code validation passed"
\end{lstlisting}
\end{tcolorbox}

\section{Remote Updates}

\begin{tcolorbox}[colback=yellow!10,colframe=orange!75!black,title=Caution: Remote Updates]
Remote update capability is essential but dangerous. Always implement:
\begin{itemize}
\item Rollback mechanism for failed updates
\item Cryptographic signature verification
\item Staged rollout (update 10\% of devices first)
\item Manual recovery method (physical button or jumper)
\end{itemize}

A failed update that bricks devices in the field will destroy your reputation overnight.
\end{tcolorbox}

\section{Scaling Challenges}

\subsection{Supply Chain Management}

As you scale beyond 1000 units per year, component sourcing becomes critical:

\begin{itemize}
\item Establish relationships with multiple suppliers
\item Monitor component lifecycle status monthly
\item Maintain 3-6 months of critical component inventory
\item Design alternative footprints for key components
\end{itemize}

\subsection{Quality Assurance}

Your test coverage must scale with production volume:

\begin{table}[h]
\centering
\begin{tabular}{|l|l|l|}
\hline
\textbf{Production Volume} & \textbf{Test Coverage} & \textbf{Sample Size} \\
\hline
1-10 units & 100\% functional test & All units \\
11-100 units & Functional + burn-in & All units \\
100+ units & Statistical sampling & 10\% + outliers \\
\hline
\end{tabular}
\caption{Recommended testing strategy by volume}
\end{table}

\section{When to Stop}

Know when to discontinue a product. Clear exit criteria prevent endless resource drain:

\begin{itemize}
\item Declining sales for 3 consecutive quarters
\item Component obsolescence requiring major redesign
\item Competitor products offering 50\% better price/performance
\item Support costs exceeding gross profit margin
\end{itemize}

The hardest decision in hardware is knowing when to stop improving and ship, and knowing when to stop shipping and move on.

\section{Building Your Next Product}

Use lessons from your first product to accelerate the second:

\begin{enumerate}
\item Reuse proven hardware blocks (power supplies, communication interfaces)
\item Maintain design consistency (connector types, mounting holes)
\item Leverage existing supplier relationships
\item Apply learned failure modes to new designs
\end{enumerate}

The second product should take 60\% of the time the first one did. The third should take 40\%. If it's taking longer, you're not learning from your mistakes.

\section{The Long Game}

Building a sustainable hardware business requires patience and persistence. Your goal is not just to ship one successful product, but to build repeatable processes that can generate multiple successful products over time.

Document everything. Every mistake, every successful solution, every supplier relationship. This knowledge becomes your competitive advantage.

Success in hardware comes not from building one perfect product, but from building an imperfect product, learning from its flaws, and systematically improving with each iteration.

The companies that win are those that can iterate fastest while maintaining quality. Speed matters, but consistency matters more.