% Chapter II
\chapter{Vision into Electronic Interfaces}

\lettrine{T}{ranslating functional requirements} into specific electronic components is where most hardware projects either succeed or fail spectacularly. This chapter walks through the systematic process of selecting components, using a time attendance system as our example.

The key is methodical evaluation: never choose the first component you find, and never assume compatibility without verification.

\section{Component Selection Process}

Let's build a time attendance system step by step, making real component choices with real tradeoffs.

\subsection{Step 1: Identify Core Components}

From our problem analysis, we need:
\begin{itemize}
\item RFID reader for user identification
\item Processing unit for database operations
\item Network interface for data synchronization
\item Local storage (non-volatile)
\item User feedback system
\end{itemize}

\subsection{Step 2: RFID Reader Selection}

Starting with RFID, I search development boards compatible with Raspberry Pi. Two modules dominate the market:

\textbf{RC522 RFID Module} - €3 from Botland (https://botland.store/rfid-modules-tags/6765-rfid-mf-rc522-module-1356mhz-spi-card-and-keychain-5904422335014.html)

\textbf{PN532 RFID Module} - €12-15 from various suppliers

\textbf{Component comparison:}

\begin{table}[h]
\centering
\begin{tabular}{|l|c|c|}
\hline
\textbf{Specification} & \textbf{RC522} & \textbf{PN532} \\
\hline
Price & €3 & €12-15 \\
Library Support (RPi) & Excellent & Limited maintenance \\
Communication & SPI & I2C/SPI/UART \\
NFC Support & No & Yes \\
Range & Standard & Slightly better \\
\hline
\end{tabular}
\caption{RFID module comparison}
\end{table}

Critical evaluation questions:

\textbf{RC522 Analysis:}
\begin{itemize}
\item \textbf{Is this chip widely manufactured?} Yes - RC522 is produced by multiple manufacturers
\item \textbf{Is it obsolete or current?} The chip is older but still in active production
\item \textbf{Library support available?} Multiple libraries exist, including my fork for CM4/CM5 kernels (https://github.com/Razikus/razrc522)
\item \textbf{Cost vs. alternatives?} €3 is competitive for 13.56MHz RFID
\end{itemize}

\textbf{PN532 Analysis:}
\begin{itemize}
\item \textbf{Additional capabilities:} NFC support, multiple communication interfaces
\item \textbf{Library maintenance:} Raspberry Pi libraries are poorly maintained, often outdated
\item \textbf{Cost impact:} 4-5x more expensive than RC522
\item \textbf{Production risk:} Limited library support increases development and maintenance burden
\end{itemize}

\textbf{Decision: RC522 selected.} For time attendance, NFC capability provides no additional value. RC522's lower cost and established library ecosystem outweigh PN532's theoretical advantages.

\subsection{Step 3: Processing Unit Selection}

For Raspberry Pi-based systems, the choice is between standard Pi boards and Compute Modules.

\textbf{My rule: Never use SD cards in production devices.} Period.

This eliminates standard Raspberry Pi boards and directs us to Compute Modules with eMMC storage.

\textbf{Compute Module 4 - Example: CM4102008}
\begin{itemize}
\item 2GB RAM, 8GB eMMC, WiFi
\item Price: ~€85 from Farnell
\item \textbf{Why 2GB RAM?} 1GB is insufficient for modern applications
\item \textbf{Why 8GB eMMC?} Adequate for OS + application + local database
\item \textbf{Why not CM5?} Excessive processing power, higher heat generation, unnecessary cost
\end{itemize}

\textbf{Critical insight:} CM5 lacks hardware H.264 encoding - video applications will consume significant CPU resources for encoding tasks.

\subsection{Step 4: Carrier Board Decision}

Two paths for CM4 integration:

\textbf{Option 1: Development Carrier Board}
Waveshare CM4-IO-BASE-C provides cameras, GPIO access, standard interfaces.

\textbf{Advantages:}
\begin{itemize}
\item Immediate availability
\item No custom PCB design required
\item Good for prototyping and small quantities
\end{itemize}

\textbf{Disadvantages:}
\begin{itemize}
\item Requires manual cable assembly
\item Not optimized for production assembly
\item Higher per-unit cost at scale
\end{itemize}

\textbf{Option 2: Custom Production Board}
Purpose-built carrier with dedicated connectors (example: \\https://shop.razniewski.eu/p/cm4pb)

\textbf{Advantages:}
\begin{itemize}
\item Optimized for production assembly
\item TE connectors for RC522, buzzer, indicators
\item Lower assembly cost per unit
\item Professional appearance
\end{itemize}

\textbf{Disadvantages:}
\begin{itemize}
\item Requires PCB design and tooling investment
\item Minimum order quantities
\item Longer development timeline
\end{itemize}

\textbf{Production volume decision threshold:} Above 100 units, custom boards become cost-effective.

\section{Component Replaceability Strategy}

\begin{tcolorbox}[colback=red!10,colframe=red!75!black,title=Critical Lesson: Component Obsolescence]
We ordered 1000 units after selling our first 100. The Ethernet connector went end-of-life between orders. Fortunately, a replacement existed, but required minor case modifications. Had we used a proprietary micro-USB connector, we would have faced complete redesign.
\end{tcolorbox}

\textbf{Replaceability guidelines:}
\begin{itemize}
\item Choose standard form factors (USB-A, RJ45, standard pin headers)
\item Avoid proprietary connectors unless absolutely necessary
\item Design footprints that accommodate multiple manufacturer variants
\item Document acceptable part substitutions in BOM
\end{itemize}

\section{Connector and Cable Strategy}

For production devices, cable assemblies determine assembly time and reliability.

\textbf{Custom Cable Manufacturing:} LCSC (https://www.lcsc.com/customcables) provides professional cable assemblies with proper strain relief and connectors.

\textbf{Example connector choice:} TE 3-640621-8 (8-pin connector)
- Standard series with multiple variants
- Available from multiple suppliers
- Established assembly processes

\textbf{Production assembly goal:} Connect RC522 module, connect I/O board, install in case, ship. Minimize field wiring and soldering.

\section{Camera Integration Considerations}

Camera selection depends on processing requirements and driver availability.

\textbf{USB Cameras with V4L2 Support}
\begin{itemize}
\item Easiest integration path
\item Standard Linux drivers
\item No custom kernel modifications required
\end{itemize}

\textbf{CSI Cameras}
\begin{itemize}
\item Higher performance potential
\item Requires Raspberry Pi-specific drivers
\item CM4 hardware acceleration available
\end{itemize}

\begin{tcolorbox}[colback=yellow!10,colframe=orange!75!black,title=Real Example: Kopin A914 Integration]
SmartHelmet project required Kopin A914 microdisplay integration via MIPI-DSI. Custom kernel driver development became necessary. The process was challenging enough with standard Linux development - on Balena platform, documentation was nonexistent, requiring extensive trial-and-error with similar drivers.
\end{tcolorbox}

\textbf{Camera selection priority:}
\begin{enumerate}
\item Standard USB with V4L2 support
\item CSI with established Raspberry Pi drivers
\item Custom hardware only when absolutely necessary
\end{enumerate}

\section{Microcontroller vs. Raspberry Pi Decision}

Not every project requires Raspberry Pi processing power.

\textbf{Use ESP32 when:}
\begin{itemize}
\item Simple data collection and transmission
\item Low power requirements critical
\item Cost optimization paramount
\item Real-time response requirements
\end{itemize}

\textbf{Use Raspberry Pi when:}
\begin{itemize}
\item Edge processing required
\item Complex user interfaces needed
\item Local database operations
\item Video/audio processing
\item Standard Linux software stack beneficial
\end{itemize}

\textbf{Time attendance system evaluation:} Requires local database, web interface, and network synchronization. Raspberry Pi appropriate choice.

\section{Power and Performance Considerations}

\textbf{CM4 vs. CM5 comparison for production:}

\begin{table}[h]
\centering
\begin{tabular}{|l|p{2.5cm}|p{2.5cm}|}
\hline
\textbf{Specification} & \textbf{CM4} & \textbf{CM5} \\
\hline
Processing Power & Adequate for most apps & Excessive for many cases \\
Heat Generation & Manageable & Requires active cooling \\
Hardware H.264 Encoding & Yes & No (CPU-based) \\
Power Consumption & Lower & Higher \\
Cost & Lower & Higher \\
\hline
\end{tabular}
\caption{CM4 vs CM5 production comparison}
\end{table}

\textbf{Recommendation:} CM4 remains optimal for most production applications unless specific CM5 features are required.

\section{Validation and Testing Strategy}

Before committing to large quantities:

\begin{enumerate}
\item \textbf{Prototype with development boards:} Verify functionality with Waveshare or similar carrier boards
\item \textbf{Test component availability:} Order small quantities from multiple suppliers
\item \textbf{Validate software libraries:} Ensure drivers work with your kernel version
\item \textbf{Thermal testing:} Verify thermal performance in your enclosure
\item \textbf{Production assembly test:} Build 5-10 units with production processes
\end{enumerate}

\section{Documentation for Production}

Create these documents before scaling:

\begin{itemize}
\item \textbf{Bill of Materials (BOM):} Include acceptable substitutions
\item \textbf{Assembly instructions:} Step-by-step with photos
\item \textbf{Test procedures:} Functional verification steps
\item \textbf{Connector pinouts:} Cable assembly documentation
\item \textbf{Software configuration:} Deployment and setup procedures
\end{itemize}

\section{Moving Forward}

With components selected and interfaces defined, the next step is building the "spider prototype" - a functional system using development boards and jumper wires to validate the complete system integration before committing to custom PCB design.

Remember: the goal is not perfect component selection, but systematic evaluation that minimizes risk while maintaining cost targets. Every component choice creates dependencies - choose components that provide multiple supply sources and established ecosystem support.