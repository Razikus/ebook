% Chapter XI
\chapter{First 100 boards}

\lettrine{T}{his is the moment} that transforms your project from an engineering exercise into a real business. You've validated your design with 5 boards, refined the software, confirmed the enclosure fit, and now you're ready to scale to 100 units. This is my favorite phase in the entire device development process—when proper preparation meets systematic execution.

If you've followed the systematic approach from previous chapters, this phase should feel almost anticlimactic. That's exactly what you want—manufacturing should be boring, predictable, and repeatable.

\section{The Beautiful Simplicity of Replication}

When everything is properly prepared, scaling to 100 units becomes a series of simple clicks:

\textbf{Step 1: JLCPCB Reorder}
Log into your JLCPCB account and find your validated PCB order. Click "Reorder" and change quantity from 5 to 100 pieces. The beauty of this moment is that everything else remains identical—same Gerber files, same components, same assembly process.

\begin{tcolorbox}[colback=yellow!10,colframe=orange!75!black,title=Component Availability Check]
Before clicking "Place Order," verify component availability. JLCPCB will flag any out-of-stock components. If substitutions are needed, this is your last chance to validate them against your tested design.
\end{tcolorbox}

\textbf{Step 2: Case Manufacturing}
Contact your enclosure partner (Cubic Inch for 3D printing or your injection molding supplier) and request 100 cases. If you've been using MJF printing, this order will likely require 2-3 weeks lead time.

\textbf{Step 3: Raspberry Pi Procurement}
Order 100 Compute Module 4 units from Farnell or your established distributor. CM4 availability can be sporadic, so order these immediately after confirming your PCB order.

\textbf{Step 4: Cable Assembly}
If using custom cables from LCSC, order 100 sets. Standard cables can be ordered from your regular suppliers.

\section{Production Organization}

Transform your workspace into a mini production line:

\subsection{Inventory Management}

\textbf{Shelf Organization Strategy:}
\begin{itemize}
\item \textbf{Shelf 1:} Incoming PCBs (sorted by batch and test status)
\item \textbf{Shelf 2:} Enclosures and mechanical components
\item \textbf{Shelf 3:} Raspberry Pi modules and accessories
\item \textbf{Shelf 4:} Cable assemblies and consumables
\item \textbf{Shelf 5:} Completed devices ready for testing
\item \textbf{Shelf 6:} Tested and packaged units ready for shipping
\end{itemize}

\textbf{Component tracking:} Use simple spreadsheet or paper checklist to track inventory consumption and identify shortages before they halt production.

\subsection{First 10 Devices Test Run}

Before committing to full production, build 10 complete devices using your production process:

\textbf{Assembly validation checklist:}
\begin{enumerate}
\item PCB-to-case fit verification (all 10 boards)
\item Connector accessibility and cable routing
\item Assembly time measurement per unit
\item Quality consistency across the batch
\item Functional testing of complete assemblies
\end{enumerate}

\textbf{Time measurement importance:} Document assembly time for these 10 units. This data determines your production capacity and labor costs for future orders.

\section{Quality Assurance at Scale}

\subsection{Testing Protocol for 100 Units}

You cannot functionally test every unit like you did with 5 prototypes. Implement statistical sampling:

\textbf{Testing strategy:}
\begin{itemize}
\item \textbf{Visual inspection:} 100\% of units (quick PCB assembly check)
\item \textbf{Power-on test:} 100\% of units (basic functionality verification)
\item \textbf{Full functional test:} 20\% of units (detailed feature validation)
\item \textbf{Burn-in testing:} 10\% of units (24-hour continuous operation)
\end{itemize}

\subsection{Defect Management}

Plan for 2-5\% defect rate even with proven components:

\textbf{Common issues at 100-unit scale:}
\begin{itemize}
\item Component placement variations
\item Solder joint quality inconsistencies
\item Case manufacturing tolerances
\item CM4 module failures (rare but possible)
\end{itemize}

\textbf{Defect resolution process:}
\begin{enumerate}
\item Document all failures with photos and symptoms
\item Categorize issues (PCB assembly vs. component vs. software)
\item Repair what's economical, scrap what's not
\item Update assembly procedures based on failure patterns
\end{enumerate}

\section{Production Metrics and Learning}

\subsection{Key Performance Indicators}

Track these metrics for your 100-unit run:

\begin{table}[h]
\centering
\begin{tabular}{|l|c|}
\hline
\textbf{Metric} & \textbf{Target} \\
\hline
Assembly time per unit & 15-30 minutes \\
First-pass yield rate & >95\% \\
Rework rate & <5\% \\
Scrap rate & <2\% \\
Software flash success & 100\% \\
\hline
\end{tabular}
\caption{Production targets for 100-unit run}
\end{table}

\textbf{Assembly time breakdown example:}
\begin{itemize}
\item CM4 installation: 2 minutes
\item Cable connections: 3 minutes
\item Enclosure assembly: 5 minutes
\item Software installation: 8 minutes
\item Quality verification: 7 minutes
\item \textbf{Total: 25 minutes per unit}
\end{itemize}

\subsection{Cost Analysis Reality Check}

Document actual costs vs. projections:

\textbf{Real cost breakdown (example):}
\begin{itemize}
\item PCB + assembly: €18.50 per unit
\item Enclosure: €12.00 per unit
\item CM4 module: €35.00 per unit
\item Cables and accessories: €8.00 per unit
\item Labor (25 min @ €20/hour): €8.33 per unit
\item \textbf{Total manufacturing cost: €81.83 per unit}
\end{itemize}

\section{Software Deployment at Scale}

\subsection{CM4 Programming Infrastructure}

Manual software installation doesn't scale to 100 units. For Compute Module 4 devices, you need reliable eMMC programming capability.

\textbf{CM4/CM5 Programmer Essential Tool:}
The CM4/CM5 Programmer (https://shop.razniewski.eu/p/cm4prog) becomes critical at this scale. This compact programmer provides professional-grade functionality for batch operations:

\begin{itemize}
\item Support for both CM4 and CM5 modules
\item Automatic eMMC boot mode switching (no manual jumpers)
\item Single USB connection for power and data
\item Convenient cutout for easy module insertion/removal
\item Designed specifically for mass production workflows
\end{itemize}

\textbf{Production programming workflow:}
\begin{enumerate}
\item Insert CM4 module into programmer
\item Switch programmer to "BOOT" mode (ON position)
\item Connect USB cable to computer
\item Run \texttt{usbboot} to detect eMMC as USB drive
\item Flash production image using standard disk imaging tools
\item Switch programmer to "RUN" mode (1 position)
\item Power cycle - module boots from flashed eMMC
\item Remove programmed module and install in carrier board
\end{enumerate}

\textbf{Automated provisioning strategy:}
Configure your production image for automatic internet registration:
\begin{itemize}
\item Include unique device identifier generation script
\item Implement first-boot internet connectivity check
\item Design automatic software download and configuration
\item Include fallback mechanism for offline scenarios
\end{itemize}

\begin{tcolorbox}[colback=blue!10,colframe=blue!75!black,title=Programming Efficiency]
With the CM4/CM5 Programmer, you can process 10-15 modules per hour including image flashing and verification. The automatic boot mode switching eliminates manual jumper manipulation, reducing errors and speeding production. For 100 units, plan 6-8 hours of dedicated programming work.
\end{tcolorbox}

\textbf{Production image design:}
Create a base image that handles device-specific configuration automatically:
\begin{itemize}
\item Generate unique device serial numbers on first boot
\item Connect to internet and register device identity
\item Download customer-specific configuration and software updates
\item Create local backup of configuration for offline operation
\item Enable remote update capability for future maintenance
\end{itemize}

\subsection{Automated Installation Process}

\textbf{Image preparation strategy:}
\begin{itemize}
\item Create master image with all software pre-installed
\item Include device-specific configuration scripts
\item Implement first-boot configuration automation
\item Test image thoroughly on multiple modules before batch programming
\end{itemize}

\begin{tcolorbox}[colback=blue!10,colframe=blue!75!black,title=Programming Efficiency]
With proper setup, the CM4 Programmer enables programming 10-15 modules per hour. For 100 units, plan a full day of dedicated programming work. Having a second programmer allows parallel operations and backup capability.
\end{tcolorbox}

\section{Common 100-Unit Production Challenges}

\subsection{Supply Chain Hiccups}

\textbf{Component shortage scenarios:}
\begin{itemize}
\item CM4 modules backordered (plan 4-6 week buffer stock)
\item JLCPCB component substitutions (validate electrical compatibility)
\item Case manufacturing delays (order cases first, longest lead time)
\item Custom cable delivery issues (maintain backup standard cable option)
\end{itemize}

\subsection{Quality Issues}

\textbf{Batch-related problems:}
\begin{itemize}
\item PCB assembly quality variation between production runs
\item Case dimensional tolerance stack-up issues
\item Component lot-to-lot performance variations
\item Software compatibility issues with newer OS versions
\end{itemize}

\begin{tcolorbox}[colback=red!10,colframe=red!75!black,title=Critical Success Factor]
The key to successful 100-unit production is having solved all fundamental problems during the 5-unit phase. Production scaling amplifies both successes and failures. Any issue that occurs 1-in-5 during prototyping will occur 20 times in your 100-unit run.
\end{tcolorbox}

\section{Preparing for the Next Scale}

\subsection{Lessons Learned Documentation}

Document everything learned during 100-unit production:

\textbf{Process improvements identified:}
\begin{itemize}
\item Assembly sequence optimizations
\item Quality check procedure refinements
\item Supplier performance evaluations
\item Cost reduction opportunities
\end{itemize}

\textbf{Design improvements for next revision:}
\begin{itemize}
\item Component placement optimizations
\item Connector accessibility improvements
\item Test point additions for production testing
\item Assembly-friendly design modifications
\end{itemize}

\subsection{Scaling Decision Point}

After successfully producing 100 units, evaluate scaling options:

\textbf{Next production volume decisions:}
\begin{itemize}
\item \textbf{200-500 units:} Refine current process, minimal tooling changes
\item \textbf{500-1000 units:} Consider injection molding for enclosures
\item \textbf{1000+ units:} Evaluate contract manufacturing partnerships
\end{itemize}

\section{The Satisfaction of Systematic Success}

The 100-unit milestone represents validation of your entire development process. When you can click "Reorder" on JLCPCB and confidently scale production 20x, you've achieved something significant.

This phase proves that hardware entrepreneurship isn't about heroic engineering efforts—it's about systematic problem-solving, careful validation, and methodical scaling. The excitement comes not from technical complexity, but from the elegant simplicity of replication.

You've transformed an idea into a repeatable manufacturing process. That's the foundation of every successful hardware business.

\textbf{Success indicators for 100-unit production:}
\begin{itemize}
\item Assembly time consistent across all units
\item Quality issues resolved through process improvements, not heroic debugging
\item Customer orders fulfilled on predictable timelines
\item Inventory management under control
\item Profit margins align with business model projections
\end{itemize}

The next chapter covers the systematic documentation of this production process, ensuring that success can be repeated and scaled further.