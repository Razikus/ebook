% Chapter VIII
\chapter{From spider into first 5 PCB}

\lettrine{Y}{our spider prototype works} and you've validated the core functionality. Now comes the transition from hand-wired chaos to professional PCB assembly. This chapter covers my process for moving from validated concept to manufactured boards.

I'm not covering PCB design itself - that's a specialized skill best left to professionals. Instead, I focus on the manufacturing process that ensures your boards arrive ready to use.

\section{The Fundamental Rule: No Manual Soldering}

\textbf{Goal: Make the process completely repeatable without your manual intervention.}

Every connector, component, and cable should be professionally assembled at the factory. Manual soldering creates bottlenecks, quality variations, and scaling problems. If you're soldering components yourself, you haven't properly industrialized your process.

Remember that every action is repeated x device you plan to manufacture. 

Sometimes is OK to soldier something if you have something simple to set up (for example soldiering 8 gold pins straight - easy peasy, even x100).

\section{JLCPCB Manufacturing Process}

I use JLCPCB (https://jlcpcb.com) for both PCB fabrication and assembly. Their integrated process provides:

\textbf{PCB + Assembly in one location:}
\begin{itemize}
\item Order PCB fabrication and component assembly together
\item Parts sourced and assembled at their facility
\item Complete boards delivered in 5 days
\item Repeatable process with consistent quality
\item Optional firmware flashing service
\end{itemize}

\textbf{Process workflow:}
\begin{enumerate}
\item Upload PCB design files (Gerbers, pick-and-place, BOM)
\item JLCPCB sources components from their inventory
\item PCB fabrication and assembly happen in parallel
\item Quality testing and packaging
\item Boards ship fully assembled and tested
\end{enumerate}

\section{Cost Considerations}

\textbf{Real JLCPCB cost example (10 boards - 2 designs):}

\begin{table}[h]
\centering
\begin{tabular}{|l|r|}
\hline
\textbf{Item} & \textbf{Cost} \\
\hline
PCB Prototype - Basic (5pcs) & \$2.01 \\
Economic PCBA - Basic (5pcs) & \$34.58 \\
PCB Prototype - Ethernet (5pcs) & \$7.06 \\
Economic PCBA - Ethernet (5pcs) & \$47.91 \\
\hline
Merchandise Total & \$91.56 \\
Shipping Charge & \$53.94 \\
Customs duties \& taxes & \$39.26 \\
\hline
\textbf{Order Total} & \textbf{\$184.76} \\
\hline
\end{tabular}
\caption{Actual JLCPCB order - 2 board variants (10 total boards)}
\end{table}

\textbf{Key observations from real pricing:}
\begin{itemize}
\item Two board designs: basic version and version with Ethernet connector
\item PCB fabrication cost minimal (\$9.07 total for both designs)
\item Assembly costs vary by complexity: \$34.58 vs \$47.91 for Ethernet version
\item Shipping (\$53.94) exceeds merchandise cost - typical for small batches
\item Duties and taxes (\$39.26) add 43\% overhead
\item \textbf{Average cost per board: \$18.48 for 10 boards across 2 designs}
\end{itemize}

\textbf{Design complexity impact:}
The Ethernet version costs 39\% more in assembly (\$47.91 vs \$34.58), demonstrating how additional components directly affect manufacturing cost. Plan component count carefully during design phase.

\section{Custom Cables Integration}

Minimize field assembly by using professional cable assemblies from LCSC Custom Cables (https://www.lcsc.com/customcables).

\textbf{Cable strategy:}
\begin{itemize}
\item Design board-to-board connections with standard connectors
\item Specify exact cable lengths and connector types
\item Include strain relief and proper jacketing
\item Order cables with boards for integrated delivery
\end{itemize}

\textbf{Result:} Final assembly becomes "connect cable A to connector B" rather than wire stripping and soldering.

\section{Design for Assembly (DFA)}

Your PCB design must accommodate automated assembly:

\textbf{Component placement considerations:}
\begin{itemize}
\item All SMD components on one side when possible
\item Adequate spacing for pick-and-place equipment
\item Through-hole components minimized or eliminated
\item Test points accessible for automated testing
\end{itemize}

\textbf{Connector strategy:}
\begin{itemize}
\item Use JLCPCB's component library when possible
\item Standard connector families (JST, Molex, TE)
\item Avoid custom or hard-to-source connectors
\item Include mounting hardware in assembly
\end{itemize}

\section{Version Control and Documentation}

\textbf{Version every PCB iteration systematically:}

\begin{tcolorbox}[colback=blue!10,colframe=blue!75!black,title=PCB Version Control Example]
\textbf{Project naming convention:}
\begin{itemize}
\item TimeAttendance\_v1.0 - Initial prototype
\item TimeAttendance\_v1.1 - Component value fixes
\item TimeAttendance\_v2.0 - Layout changes
\item TimeAttendance\_v2.1 - Manufacturing optimization
\end{itemize}

\textbf{Cloud storage structure:}
\begin{verbatim}
/TimeAttendance_PCB/
  /v1.0/
    /design_files/
    /assembly_docs/
    /test_results/
  /v1.1/
    /design_files/
    /assembly_docs/
    /test_results/
\end{verbatim}
\end{tcolorbox}

\textbf{Documentation requirements for each version:}
\begin{itemize}
\item Complete design files (schematic, layout, Gerbers)
\item Bill of Materials with part numbers
\item Assembly drawings and instructions
\item Test procedures and acceptance criteria
\item Change log from previous version
\end{itemize}

\section{First 5 PCB Strategy}

Order exactly 5 boards for your first iteration:

\textbf{Board allocation:}
\begin{itemize}
\item \textbf{Board 1:} Immediate functional testing
\item \textbf{Board 2:} Software development platform
\item \textbf{Board 3:} Environmental stress testing
\item \textbf{Board 4:} Customer demonstration unit
\item \textbf{Board 5:} Archive/backup for troubleshooting
\end{itemize}

This quantity provides adequate testing capability without excessive cost for changes that will inevitably be needed.

\section{Quality Validation Process}

\textbf{Upon receiving boards, systematic testing:}

\begin{enumerate}
\item \textbf{Visual inspection:} Component placement, solder quality, damage
\item \textbf{Power-on test:} Verify power rails and basic functionality
\item \textbf{Interface testing:} All connectors and communication buses
\item \textbf{Software deployment:} Load and test application software
\item \textbf{Performance validation:} Compare to spider prototype benchmarks
\end{enumerate}

\textbf{Pass criteria:} At least 4 of 5 boards must pass all tests. If fewer pass, investigate root cause before ordering next iteration.

\section{Common First PCB Issues}

\textbf{Expect these problems on first iteration:}
\begin{itemize}
\item Component footprint mismatches
\item Pin assignment errors
\item Inadequate power supply decoupling
\item Missing pull-up/pull-down resistors
\item Connector pinout mistakes
\end{itemize}

\textbf{Design for debugging:}
\begin{itemize}
\item Include test points for all power rails
\item Add LED indicators for major functions
\item Provide jumpers for configuration changes
\item Include serial debug connector
\end{itemize}

\section{Preparing for Next Iteration}

Document everything learned from first PCB batch:

\textbf{Technical changes needed:}
\begin{itemize}
\item Component value adjustments
\item Layout improvements
\item Additional features required
\item Manufacturing feedback incorporation
\end{itemize}

\textbf{Process improvements:}
\begin{itemize}
\item Assembly time reduction opportunities
\item Test procedure refinements
\item Documentation clarity issues
\item Cost optimization possibilities
\end{itemize}

The first PCB iteration validates electrical design and manufacturing processes. Expect significant changes before the second iteration, which will focus on mechanical integration and production optimization.

\section{Success Metrics}

\textbf{First PCB iteration is successful if:}
\begin{itemize}
\item Basic functionality matches spider prototype performance
\item Manufacturing process is repeatable and documented
\item Assembly time is predictable and reasonable
\item Cost projections align with business model
\item Quality issues are identifiable and correctable
\end{itemize}

The goal isn't perfection - it's systematic validation of your transition from prototype to manufactured product. Each iteration should solve specific, documented problems while maintaining overall functionality.