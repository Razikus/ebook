% Chapter IV
\chapter{Build a "spider" prototype}

\lettrine{Y}{ou've analyzed the problem} and selected components. Now comes the critical validation phase: proving your concept actually works before committing to PCB design and production tooling.

This is the transition from TRL 3 (experimental proof of concept) to TRL 4 (technology validated in lab). The "spider" prototype gets its name from the web of jumper wires connecting components - it looks chaotic but serves a crucial purpose.

\section{The Technology Readiness Level Context}

Understanding where you are in the development process prevents premature optimization:

\textbf{TRL 1-2:} Basic principles and concept formulation (completed in analysis phase)
\textbf{TRL 3:} Experimental proof of concept (what we're building now)
\textbf{TRL 4:} Technology validated in lab environment (our goal for this chapter)
\textbf{TRL 5-6:} Validation in relevant/operational environment (future chapters)
\textbf{TRL 7-9:} System demonstration and proven operation (production phase)

\section{Spider Prototype Objectives}

The spider prototype must validate every critical function:

\textbf{For our time attendance system:}
\begin{itemize}
\item RFID reader successfully captures UID
\item CM4 maintains stable internet connection
\item System executes REST API calls reliably
\item Buzzer provides audio feedback
\item GPIO controls function correctly
\item Data persistence works (file or database)
\item Web interface displays collected data
\end{itemize}

\section{Minimal Software Requirements}

Write the simplest possible code that exercises every function:

\begin{tcolorbox}[colback=gray!5,colframe=gray!50,title=Basic Validation Script]
\begin{lstlisting}[language=Python,basicstyle=\small\ttfamily]
import time
import requests
from razrc522.rfid import RFID
from razrc522.easyrfid import EasyRFID, EasyRFIDUIDMode, EasyRFIDAuth
reader = RFID(antenna_gain=7, logger=None)
easyRFID = EasyRFID(reader, mode=EasyRFIDUIDMode.HEX)
def test_network():
    try:
        response = requests.get('https://httpbin.org/get', timeout=5)
        return response.status_code == 200
    except:
        return False
def read_rfid_card():
    try:
        convertedUID, rawUID = easyRFID.wait_and_select()
        block = 0
        # Standard MIFARE 1k key
        authorized = easyRFID.authorize(EasyRFIDAuth.AuthB, 
                                      [0xFF, 0xFF, 0xFF, 0xFF, 0xFF, 0xFF], 
                                      rawUID, block)
        if authorized:
            readed = easyRFID.read_block(block)
            ....
...
\end{lstlisting}
\end{tcolorbox}

\section{Critical Validation Questions}

Run your spider prototype and systematically evaluate:

\subsection{Basic Functionality}
\begin{itemize}
\item \textbf{Does it work at all?} Can you complete the full cycle without errors?
\item \textbf{How cumbersome is interaction?} Does RFID reading require multiple attempts?
\item \textbf{Response time:} How fast does the system respond to card presentation?
\item \textbf{Reliability:} Test the same operation 50 times - how many succeed?
\end{itemize}

\subsection{Environmental Factors}
\begin{itemize}
\item \textbf{Heat generation:} Does the CM4 require cooling under normal operation?
\item \textbf{Power consumption:} Measure actual current draw vs. specifications
\item \textbf{Voltage compatibility:} Verify 3.3V GPIO levels work with your peripherals
\item \textbf{EMI sensitivity:} Do nearby phones or WiFi networks cause interference?
\end{itemize}

\subsection{Persistence Testing}
\begin{itemize}
\item \textbf{24-hour test:} Does the system work reliably after running overnight?
\item \textbf{1-hour stress test:} Continuous operation with frequent card reads
\item \textbf{Network interruption:} How does the system behave when WiFi disconnects?
\item \textbf{Power cycle recovery:} Does it restart cleanly after power loss?
\end{itemize}

\begin{tcolorbox}[colback=red!10,colframe=red!75!black,title=Critical Voltage Warning]
Raspberry Pi GPIO operates at 3.3V, not 5V. Connecting 5V signals to GPIO pins will damage your CM4. Always verify peripheral voltage requirements and use level shifters when necessary.
\end{tcolorbox}

\section{Spider Prototype Assembly}

\textbf{Option 1: Individual Components}

\textbf{Required components for time attendance validation:}
\begin{itemize}
\item CM4 on development carrier board (Waveshare CM4-IO-BASE-C or similar)
\item RC522 RFID module
\item Buzzer (3.3V compatible)
\item LED for status indication
\item Breadboard and jumper wires
\item RFID cards/tags for testing
\end{itemize}

\textbf{Connection diagram:}
\begin{verbatim}
RC522 -> CM4 GPIO
SDA   -> GPIO 8  (SPI CE0)
SCK   -> GPIO 11 (SPI CLK)
MOSI  -> GPIO 10 (SPI MOSI)
MISO  -> GPIO 9  (SPI MISO)
RST   -> GPIO 25
3.3V  -> 3.3V
GND   -> GND

Buzzer -> GPIO 18
LED    -> GPIO 24 (with current limiting resistor)
\end{verbatim}

\textbf{Option 2: Complete RFID Solution}

For faster prototyping, use the complete CM4 RFID board (https://shop.razniewski.eu/p/cm4pb) which includes CM4, RFID reader, and all necessary interfaces pre-integrated. This eliminates wiring complexity and reduces validation time to software testing only.

\section{Iteration and Refinement}

Spider prototypes exist to fail safely. Expect multiple iterations:

\textbf{Common first iteration failures:}
\begin{itemize}
\item RFID reading inconsistent (adjust antenna positioning)
\item Network timeouts (implement retry logic)
\item GPIO conflicts (verify pin assignments)
\item Power supply inadequate (check current requirements)
\end{itemize}

\textbf{Performance optimization targets:}
\begin{itemize}
\item RFID read success rate > 95\%
\item Response time < 2 seconds
\item Network operation timeout < 5 seconds
\item System uptime > 24 hours without intervention
\end{itemize}

\textbf{Material interference testing:}
Test RFID scanning through different materials, particularly those you plan to use for the final enclosure:
\begin{itemize}
\item Plastic enclosure materials (ABS, polycarbonate)
\item Metal proximity effects (mounting screws, brackets)
\item Glass or acrylic cover materials
\item Varying material thicknesses
\end{itemize}

\section{Component Validation Results}

After spider prototype testing, you'll know definitively:

\textbf{Keep these components if:}
\begin{itemize}
\item Functionality works reliably in spider configuration
\item Performance meets your requirements
\item Heat generation is manageable
\item Cost targets are achievable
\end{itemize}

\textbf{Reconsider components if:}
\begin{itemize}
\item Reliability issues persist after optimization attempts
\item Performance significantly below expectations
\item Integration complexity exceeds development budget
\item Alternative components offer substantial advantages
\end{itemize}

\section{Decision Point: PCB or Software First}

With validated spider prototype, choose your next development path:

\textbf{PCB Design First:}
\begin{itemize}
\item Hardware functionality proven stable
\item Mechanical constraints well understood
\item Production timeline pressure
\item External PCB design resources available
\end{itemize}

\textbf{Software Development First:}
\begin{itemize}
\item Complex software requirements
\item User interface needs refinement
\item Integration with external systems required
\item Software team available in-house
\end{itemize}

\textbf{My typical approach:} Software development while PCB design proceeds in parallel. Software can be developed and tested on spider prototype while PCB layout is completed by external partners.

\section{External PCB Design Considerations}

Most hardware entrepreneurs benefit from partnering with PCB design specialists rather than learning complex layout tools.

\textbf{First PCB reality check:}
Your first PCB will not be ready for final enclosure integration. Accept this from the start.

At this point, consider your enclosure strategy by reviewing the Appendix resources:
\begin{itemize}
\item \textbf{Kradex:} Standard enclosures for cost-effective solutions
\item \textbf{Cubic Inch:} Custom 3D printed enclosures for unique requirements
\end{itemize}

\textbf{Think about placement and constraints, but don't optimize yet:}
\begin{itemize}
\item How will components be positioned relative to each other?
\item What connectors need external access?
\item Where will status indicators be visible?
\item How will the RFID antenna be oriented?
\end{itemize}

Mechanical constraints will change as you iterate. The PCB will evolve significantly between first prototype and production version.

\textbf{Prepare for external PCB design:}
\begin{itemize}
\item Detailed pinout documentation from spider prototype
\item Basic mechanical requirements (board size, connector positions)
\item Performance requirements (signal integrity, EMI considerations)
\item Manufacturing constraints (assembly capabilities, test points)
\end{itemize}

\textbf{Iteration expectation:} Plan for 2 PCB iterations minimum. The first validates electrical functionality. The second optimizes for mechanical integration and manufacturing.

\section{Moving to Implementation}

Your spider prototype has validated the core technology. You now have:
\begin{itemize}
\item Proven component functionality
\item Working software foundation
\item Performance baselines
\item Integration challenges identified
\item Development path clarity
\end{itemize}

The next phase involves either developing production software on your proven hardware platform or translating the spider prototype into a proper PCB design that maintains the validated functionality while enabling efficient manufacturing.

Remember: the spider prototype's job is to fail fast and fail cheap. Every failure discovered here saves weeks of debugging production hardware later.