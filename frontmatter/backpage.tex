\pagestyle{empty}  % Disable headers and footers on this page

\cleardoublepage    % Ensure the following content starts on a right-hand page in double-sided layouts

% =================
% Back cover design
% =================
\begin{center}
\begin{tikzpicture}[remember picture,overlay]

    % White background
    \fill[white] (current page.south west) rectangle (current page.north east);
    
    % Simple black border (matching front cover)
    \draw[black, line width=2pt] 
        ([xshift=20pt, yshift=20pt] current page.south west) rectangle 
        ([xshift=-20pt, yshift=-20pt] current page.north east);

    % Book description area
    \node[black, text width=0.8\linewidth, align=justify, anchor=north] 
        (description) at ([yshift=-80pt] current page.north) {
        \textbf{Learn to build production-ready devices with Raspberry Pi Compute Module}
        
        \vspace{12pt}
        
        This practical guide takes you from prototype to market-ready product. Based on real experience from building and selling over 5,000 devices worldwide.
        
        \vspace{12pt}
        
        \textbf{What you'll learn:}
        \begin{itemize}
        \item Convert your vision into electronic interfaces
        \item Choose the right peripherals and avoid costly mistakes
        \item Build reliable prototypes and production PCBs
        \item Write production-ready software with remote updates
        \item Navigate certification processes
        \item Scale from prototype to first 100 devices
        \end{itemize}
        
        \vspace{12pt}
        
        \textbf{Perfect for:} Engineers, makers, and entrepreneurs who want to turn their Raspberry Pi projects into commercial products.
    };

    % Simple Raspberry Pi circuit pattern (bottom center)
    \begin{scope}[remember picture, overlay, shift={($(current page.south)+(0,120pt)$)}]
        % Circuit traces
        \draw[black, line width=0.5pt] (-2,0) -- (2,0);
        \draw[black, line width=0.5pt] (0,-1) -- (0,1);
        \draw[black, line width=0.5pt] (-1.5,-0.5) -- (1.5,0.5);
        \draw[black, line width=0.5pt] (-1.5,0.5) -- (1.5,-0.5);
        
        % Small circles representing components
        \foreach \x/\y in {-1.5/0, -0.75/0.5, 0/0, 0.75/-0.5, 1.5/0} {
            \fill[black] (\x,\y) circle (0.05);
        }
    \end{scope}

    % Author name (bottom right)
    \node[black, anchor=south east] 
        at ([xshift=-40pt, yshift=40pt] current page.south east) 
        {\textsc{\authorname}};

    % Publisher name (bottom left)
    \node[black, anchor=south west] 
        at ([xshift=40pt, yshift=40pt] current page.south west) 
        {\textsc{\publisher}};

\end{tikzpicture}
\end{center}