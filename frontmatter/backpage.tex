\pagestyle{empty}

\cleardoublepage

% =================
% Back cover design
% =================
\begin{center}
\begin{tikzpicture}[remember picture,overlay]

    % White background
    \fill[white] (current page.south west) rectangle (current page.north east);
    
    % Simple black border (matching front cover)
    \draw[black, line width=2pt] 
        ([xshift=20pt, yshift=20pt] current page.south west) rectangle 
        ([xshift=-20pt, yshift=-20pt] current page.north east);

    % Book description area
    \node[black, text width=0.75\linewidth, align=justify, anchor=north] 
        (description) at ([yshift=-90pt] current page.north) {
        \textbf{From prototype to 100+ commercial devices in production}
        
        \vspace{12pt}
        
        This systematic guide provides frameworks for building hardware businesses with Raspberry Pi. Based on real experience from shipping over 5,000 devices worldwide.
        
        \vspace{12pt}
        
        \textbf{Learn the systematic approach to:}
        \begin{itemize}
        \item Analyze problems and select components methodically
        \item Build "spider" prototypes that validate concepts
        \item Design for automated PCB assembly (no manual soldering)
        \item Scale manufacturing from 5 to 100+ units
        \item Plan production-ready software architecture
        \item Navigate certification and supplier relationships
        \end{itemize}
        
        \vspace{12pt}
        
        \textbf{For engineers and entrepreneurs ready to move beyond hobby projects.}
    };

    % Simple Raspberry Pi circuit pattern (bottom center)
    \begin{scope}[remember picture, overlay, shift={($(current page.south)+(0,120pt)$)}]
        % Circuit traces
        \draw[black, line width=0.5pt] (-2,0) -- (2,0);
        \draw[black, line width=0.5pt] (0,-1) -- (0,1);
        \draw[black, line width=0.5pt] (-1.5,-0.5) -- (1.5,0.5);
        \draw[black, line width=0.5pt] (-1.5,0.5) -- (1.5,-0.5);
        
        % Small circles representing components
        \foreach \x/\y in {-1.5/0, -0.75/0.5, 0/0, 0.75/-0.5, 1.5/0} {
            \fill[black] (\x,\y) circle (0.05);
        }
    \end{scope}

    % Author name (bottom right)
    \node[black, anchor=south east] 
        at ([xshift=-40pt, yshift=40pt] current page.south east) 
        {\textsc{\authorname}};

    % Publisher name (bottom left)
    \node[black, anchor=south west] 
        at ([xshift=40pt, yshift=40pt] current page.south west) 
        {\textsc{\publisher}};

\end{tikzpicture}
\end{center}